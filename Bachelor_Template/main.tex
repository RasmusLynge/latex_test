\documentclass{book}
\usepackage[utf8]{inputenc}

\usepackage{graphicx}
\graphicspath{ {./images/} }
\usepackage{caption}
\usepackage{subcaption}
\usepackage{enumitem}%
\usepackage[T1]{fontenc}
\usepackage{inconsolata}

\usepackage{color}

\newcommand\todo[1]{\textcolor{red}{#1}}

\definecolor{pblue}{rgb}{0.13,0.13,1}
\definecolor{pgreen}{rgb}{0,0.5,0}
\definecolor{pred}{rgb}{0.9,0,0}
\definecolor{pgrey}{rgb}{0.46,0.45,0.48}

\usepackage{listings}
\lstset{language=Java,
  showspaces=false,
  showtabs=false,
  breaklines=true,
  showstringspaces=false,
  breakatwhitespace=true,
  commentstyle=\color{pgreen},
  keywordstyle=\color{pblue},
  stringstyle=\color{pred},
  basicstyle=\ttfamily,
  moredelim=[il][\textcolor{pgrey}]{$$},
  moredelim=[is][\textcolor{pgrey}]{\%\%}{\%\%}
}


\title{Bachelor Template}
\author{Rasmus Lynge Jacobsen}
\date{March 2021}




\begin{document}

\maketitle
\begingroup
\let\cleardoublepage\clearpage
\tableofcontents
\endgroup


\chapter{Images}
\section{Images with captions}

\begin{figure}[h]
\centering
\includegraphics[width=0.25\textwidth]{pepe.jpg}
\caption{Caption under image}
\label{fig:pepe1}
\end{figure}

\begin{figure}[h]
\centering
\caption{Caption over image}
\includegraphics[width=0.25\textwidth]{pepe.jpg}
\end{figure}


\begin{figure}%
    \centering
    \subfloat[\centering one picture]{{\includegraphics[width=5cm]{images/pepe.jpg} }}%
    \qquad
    \subfloat[\centering two picture]{{\includegraphics[width=5cm]{images/pepe.jpg} }}%
    \caption{Two images next to each other}%
    \label{fig:example}%
\end{figure}

\section{Image reference}
\ref{fig:pepe1} This is a reference to the 1st pepe picture. \\
See figure~\ref{fig:pepe1} on page~\pageref{fig:pepe1}

\chapter{Document Sectioning}
\section{Section}
\subsection{Subsection}
\paragraph{Paragraph}
\subparagraph{Subparagraph}
\section*{Non-numbered section}

\chapter{Lists}
\section{Bullet points}
\begin{itemize}
  \item One entry in the list
  \item Another entry in the list
\end{itemize}

\section{Alternative bullet symbols}
\begin{enumerate}[label={\includegraphics[width=0.1\textwidth]{pepe.jpg}}]
  \item First item
  \item Second item
  \item \ldots
  \item Last item
\end{enumerate}

\section{ Enumerated lists}
\begin{enumerate}
  \item One entry in the list
  \item Another entry in the list
\end{enumerate}

\chapter{Tables}
\section{Table with multiple columns}
\begin{center}
\begin{tabular}{ |c|c|c| } 
 \hline
 cell1 & cell2 & cell3 \\ 
 cell4 & cell5 & cell6 \\ 
 cell7 & cell8 & cell9 \\ 
 \hline
\end{tabular}
\end{center}

\section{Table with description and label}
The table \ref{table:1} is an example of referenced \LaTeX elements.

\begin{table}[h!]
\centering
\begin{tabular}{||c c c c||} 
 \hline
 Col1 & Col2 & Col2 & Col3 \\ [0.5ex] 
 \hline\hline
 1 & 6 & 87837 & 787 \\ 
 2 & 7 & 78 & 5415 \\
 3 & 545 & 778 & 7507 \\
 4 & 545 & 18744 & 7560 \\
 5 & 88 & 788 & 6344 \\ [1ex] 
 \hline
\end{tabular}
\caption{Table to test captions and labels}
\label{table:1}
\end{table}

\ref{table:1} This is a table ref

\chapter{Code}


\begin{lstlisting}

package dk.mmmr.math.interfaces;

/**
 * This is a doc comment.
 */
public interface StringSorter {
    String[] sort(String[] arr);
    default void printArray(String[] arr)
    {
        String[] toPrint =  arr;
        for (int i = 0; i < toPrint.length; i++)
            System.out.print(arr[i] + " ");
    }
}
\end{lstlisting}

\chapter{Math}
\section{Inline equations}
The well known Pythagorean theorem \(x^2 + y^2 = z^2\) was 
proved to be invalid for other exponents. 
\section{Display equations}
Meaning the next equation has no integer solutions:

\[ x^n + y^n = z^n \]
\section{Fractions, summations, products, roots, powers}
\begin{itemize}
  \item Fraction = \(\frac{3x}{2}\) 
  \item Summation = $\sum_{n=1}^{\infty} 2^{-n} = 1$
  \item Product = $\prod_{i=a}^{b} f(i)$
  \item Roots = \(\sqrt[34]{test}\)
  \item Powr = \(test^2\)
\end{itemize}


\chapter{Todo?}
\todo{TODO}

\begin{thebibliography}{9}
\bibitem{latexcompanion} 
Michel Goossens, Frank Mittelbach, and Alexander Samarin. 
\textit{The \LaTeX\ Companion}. 
Addison-Wesley, Reading, Massachusetts, 1993.

\bibitem{einstein} 
Albert Einstein. 
\textit{Zur Elektrodynamik bewegter K{\"o}rper}. (German) 
[\textit{On the electrodynamics of moving bodies}]. 
Annalen der Physik, 322(10):891–921, 1905.

\bibitem{knuthwebsite} 
Knuth: Computers and Typesetting,
\\\texttt{http://www-cs-faculty.stanford.edu/\~{}uno/abcde.html}
\end{thebibliography}


\end{document}
