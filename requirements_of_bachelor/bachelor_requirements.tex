\documentclass{article}
\usepackage[utf8]{inputenc}
\usepackage{csquotes}

\title{Requirements of bachelor thesis}
\author{Rasmus Jacobsen}

\begin{document}

\maketitle

\section{Generelle regler for eksamen}

\begin{quote}
For prøver og eksamen på Cphbusiness gælder reglerne i BEK 1519 af 16/12/2013:
Bekendtgørelse om prøver i erhvervsrettede videregående uddannelser og BEK 262 af 20/03/2007: Bekendtgørelsen om karakterskala og anden bedømmelse. Derudover
gælder den senest offentliggjorte version af Cphbusiness’ eksamensreglement, uddannelsesspecifikke eksamensregelsæt og eksamensmanualer.

\end{quote}

\subsection{Eksamensform}
\begin{quote}
Eksamen i bachelorprojektet afvikles som en ekstern prøve, som sammen med prøven efter praktikken og uddannelsens øvrige prøver skal dokumentere, at uddannelsens mål for læringsudbytte er opnået. Prøven består i et skriftligt projekt og en
mundtlig del, hvor der gives én samlet karakter. Prøven kan først finde sted efter, at
afsluttende prøve i praktikken og uddannelsens øvrige prøver er bestået.
\end{quote}

\section{Krav til bachelorprojektet}

\begin{quote}
Bachelorprojektet skal dokumentere den studerendes forståelse af praksis og centralt
anvendt teori og metode i relation til en praksisnær problemstilling, der tager udgangspunkt i en konkret opgave inden for uddannelsens område. Problemstillingen,
der således skal være central for uddannelsen og erhvervet, formuleres af den studerende, eventuelt i samarbejde med en privat eller offentlig virksomhed. Cphbusiness
godkender problemstillingen.
\end{quote}

\subsection{Tidsmæssig placering}
3. semester
\subsection{Omfang}
15 ECTS

\subsection{Indhold}
\begin{quote}
I bachelorprojektet skal den studerende dokumentere evnen til på et
analytisk og metodisk grundlag at kunne bearbejde en kompleks og praksisnær
problemstilling i relation til en konkret opgave inden for IT-området. 
\end{quote}


\subsection{Læringsmål}
Det afsluttende bachelorprojekt skal dokumentere, at uddannelsens afgangsniveau er opnået, jf. bilag 1 i BEK for professionsbacheloruddannelsen i softwareudvikling:
\subsubsection{Viden}
Den uddannede har viden om:
\begin{itemize}
  \item den strategiske rolle af test i systemudvikling
  \item globalisering af softwareproduktion
  \item systemarkitektur og forståelse af dens strategiske betydning for virksomhedens forretning
  \item anvendt teori og metode samt udbredte teknologier inden for domænet
  \item sammenhænge mellem anvendt teori, metode og teknologi og kan reflektere over disses egnethed i forskellige situationer
\end{itemize}
\subsubsection{Færdigheder}
Den uddannede kan:
\begin{itemize}
\item håndtere planlægning og gennemførelse af test af større IT-systemer
\item indgå professionelt i samarbejde omkring udvikling af store systemer ved anvendelse af udbredte metoder og teknologier
\item sætte sig ind i nye teknologier og standarder til håndtering af integration mellem systemer
\item gennem praksis udvikle egen kompetenceprofil fra primært at være en
backend-udviklerprofil til at varetage opgaver som systemarkitekt
\item gennem praksis udvikle egen kompetenceprofil fra primært at være en
backend-udviklerprofil til at varetage opgaver som systemarkitekt
\item håndtere fastlæggelse og realisering af en såvel forretningsmæssig som
teknologisk hensigtsmæssig arkitektur for store systemer
\end{itemize}

\subsubsection{Kompetencer}
Den uddannede kan:
\begin{itemize}
    \item håndtere planlægning og gennemførelse af test af større IT-systemer
    \item indgå professionelt i samarbejde omkring udvikling af store systemer ved anvendelse af udbredte metoder og teknologier
    \item indgå professionelt i samarbejde omkring udvikling af store systemer ved anvendelse af udbredte metoder og teknologier
    \item sætte sig ind i nye teknologier og standarder til håndtering af integration mellem systemer
    \item gennem praksis udvikle egen kompetenceprofil fra primært at være en
    backend-udviklerprofil til at varetage opgaver som systemarkitekt
    \item håndtere fastlæggelse og realisering af en såvel forretningsmæssig som teknologisk hensigtsmæssig arkitektur for store systemer
\end{itemize} 


\begin{thebibliography}{9}
\bibitem{Studieordning for softwareudvikling} 
\textit{Studieordning for softwareudvikling} \\
Udarbejdet iht. BEK nr. 1521 af 16/12/2013: Bekendtgørelse om erhvervsakademiuddannelser og professionsbacheloruddannelser af de institutioner, som er godkendt til udbud af uddannelsen.

\texttt{https://www.cphbusiness.dk/media/1177/pba_soft_cba_studieordning.pdf}

\end{thebibliography}
\end{document}
